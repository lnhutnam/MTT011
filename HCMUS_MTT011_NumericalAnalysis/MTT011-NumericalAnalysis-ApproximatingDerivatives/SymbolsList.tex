\chapter*{DANH MỤC CÁC KÝ HIỆU, CHỮ VIẾT TẮT}
\addcontentsline{toc}{chapter}{{\bf DANH MỤC CÁC KÝ HIỆU, CHỮ VIẾT TẮT}}
\begin{center}
	\begin{longtable}{ l  l }
		$\mathcal{R}$ &Ngưỡng lợi ích tối thiểu \\
		CPU& Bộ xử lý trung tâm\\
		&(Central processing unit)\\
		CSP& Bài toán thỏa mãn ràng buộc\\
		&(Constraints satisfaction problem)\\
		GPU& Bộ xử lý đồ họa\\
		&(Graphic processing unit)\\
		HTWUI& Itemset TWU cao\\
		&(High-TWU itemset)\\
		HUI & Itemset lợi ích cao\\
		&(High-utility itemset)\\
		HUPM & Khai thác mẫu lợi ích cao\\
		&(High-utility pattern mining)\\
		ILP & Quy hoạch số nguyên tuyến tính\\
		& (Integer linear programming)\\
		IP & Quy hoạch số nguyên\\
		&(Integer programming)\\
		LP& Quy hoạch tuyến tính\\
		&(Linear programming)\\
		LP relaxation& Lời giải nới lỏng của bài toán quy hoạch số nguyên\\
		LUI&Itemset lợi ích thấp\\
		&(Low-utility itemset)\\
		SHUI & Itemset lợi ích cao nhạy cảm\\
		&(Sensitive high-utility itemset)\\
		NSHUI &Itemset lợi ích cao không nhạy cảm \\
		&(Non-sensitive high-utility itemset)\\
		PPFIM& Bảo vệ tính riêng tư trong khai thác tập phổ biến\\
		& Privacy-preserving frequent itemset mining\\
		PPUM& Bảo vệ tính riêng tư trong khai thác mẫu hữu ích\\
		&(Privacy-preserving utility mining)\\
		SIP& Tỷ lệ thông tin nhạy cảm\\
		&(Sensitive information percentage)\\
\end{longtable}

\end{center}