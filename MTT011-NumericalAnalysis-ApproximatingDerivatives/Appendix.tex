\appendix

\chapter*{PHỤ LỤC 1: NỘI SUY HERMITE}
\addcontentsline{toc}{chapter}{{\bf PHỤ LỤC 1: NỘI SUY HERMITE}}

Giả định rằng tại một điểm cho trước $x_j$, ta biết một giá trị hàm $y_i$, và có thể có nhiều giá trị đạo hàm $y_j', y''_j, \dots, y_j^{m_j}$. Ta có thể sử dụng ma trận Vandermonde để chứng tỏ rằng tồn tại một đa thức duy nhất $P$ với bậc không nhỏ hơn $d$ mà thỏa mãn
\begin{equation}
    P^{(k)}(x_j) = y^{(k)}_j, k = 0, 1, \dots, m_j, j = 0, 1, \dots, n
\end{equation}
trong đó $d = n + m_0 + m_1 + \dots + m_m$. Đa thức osculating (kissing) là một đa thức nội suy rất tổng quát.

Thật vậy, ta có thể đặt tên cho loại đa thức xấp xỉ được gắn với mỗi loại dữ liệu và cho bậc của chúng.
\begin{enumerate}
    \item $n >0$, $m_j = 0$, $j = 0, 1, \dots, n$
    \item $n >0$, $m_j = 1$, $j = 0, 1, \dots, n$
    \item $n = 0$, $n_0 = N$
\end{enumerate}

Khối Vandermonde liên kết với thành phần $j$ bây giờ có $1 + m_j$ dòng, và
\begin{equation}
    \begin{bmatrix}
         1&  x&  x^2&  x^3& \dots & x^d\\ 
         0&  1&  2x&  3x^2& \dots&dx^{d-1}\\ 
         0&  0&  2&  6x&\dots &d(d-1)x^{d-2}\\ 
         &  &  &   & \vdots&
    \end{bmatrix}
\end{equation}
được đánh giá tại $x = x_j$. Ma trận Vandermonde $(d+1) \times (d+1)$ này là không suy biến nếu và chỉ nếu $x_j$ phân biệt.

Trong một số ứng dụng, $y_i$ là những giá trị của một hàm đã biết $f$. Nếu $f \in C^{d + 1}([a, b])$, và $[a, b]$ chứa tất cả các điểm, thì $\forall x \in [a, b], \exists\xi \in [a,b]$ sao cho
\begin{equation}
    f(x) = P(x) + \frac{f^{(d+1)}(\xi)}{(d+1)!}\prod_{j=0}^n(x-x_j)^{m_j+1}
\end{equation}

Việc sử dụng một đa thức để nội suy một tập hợp lớn các điểm (hoặc một số lượng lớn các điều kiện trên một tập hợp điểm) đòi hỏi bộ nội suy phải có bậc lớn. Điều này thường tạo ra các dao động lớn và không mong muốn trong bộ nội suy và khiến việc đánh giá $P(x)$ tốn kém hơn. Nếu ta có quyền tự do lựa chọn các nút, chúng có thể được chọn để giảm thiểu các dao động dữ dội; đây được gọi là lựa chọn nút Chebyshev.

\chapter*{PHỤ LỤC 2: THIẾT LẬP CÁC CHẶN SAI SỐ CHO NỘI SUY HERMITE}
\addcontentsline{toc}{chapter}{{\bf PHỤ LỤC 2: THIẾT LẬP CÁC CHẶN SAI SỐ CHO NỘI SUY HERMITE}}


\section{Chứng minh 2}


\chapter*{PHỤ LỤC 3: LÝ THUYẾT CƠ BẢN VỀ XẤP XỈ ĐA THỨC}
\addcontentsline{toc}{chapter}{{\bf PHỤ LỤC 3: LÝ THUYẾT CƠ BẢN VỀ XẤP XỈ ĐA THỨC}}

Xem xét một không gian tuyến tính (linear space) không nhất thiết phải là không gian hữu hạn chiều (finte dimensional space) mà các phần tử của nó là các hàm $\{f(x)\}$. Chuẩn (norm) được định nghĩa là một phép gán một số thực vào mỗi phần tử của không gian tuyến tính trên, ký hiệu $\text{Norm}(f) \equiv N(f) \equiv \left \|  f\right \|$ thỏa mãn:
\begin{itemize}
    \item $\left \|  f\right \| \geq 0$,
    \item $\left \|  f\right \| = 0$ nếu và chỉ nếu $f(x) \equiv 0$,
    \item $\left \|  cf\right \| =  \left | c \right |\cdot \left \|  f\right \|$, với mọi hằng số $c \in \R$,
    \item $\left \|  f + g\right \| \leq \left \|  f\right \| + \left \|  g\right \|$
\end{itemize}

Một độ đo của độ lệch chuẩn (measure of the deviation) hay độ lỗi (error) trong một xấp xỉ của $f(x)$ bởi $P_n(x)$ được định nghĩa một cách tổng quát bởi khái niệm bán chuẩn (semi-norm):
\begin{equation}
    \left | f(x) - P_n(x) \right |_{sn}
\end{equation}

Ta giả định rằng các đa thức và hàm, $f(x)$, cần được xấp xỉ trong một không gian tuyến $C[a, b]$ của các hàm đươc định nghĩa trên một khoảng bị chặn đóng, $[a, b]$. 

\begin{theorem}
    \label{theorem:3.0.0}
    Gọi một độ đo của độ lệch chuẩn $\left | . \right |_{sn}$ được định nghĩa trong $C[a, b]$, và tồn tại các số dương $m_n$ và $M_n$ thỏa mãn:
    \begin{equation}
        0 < m_n \leq \left | \sum_{j=0}^nb_jx^j \right |_{sn} \leq M_n, n = 0, 1, \dots
    \end{equation}
    với mọi $\{b_j\}$ mà thỏa mãn
    \begin{equation}
        \sum_{j=0}^nb_j = 1
    \end{equation}
    Thì với bất kỳ số nguyên $n$ và $f(x)$ trong $C[a, b]$, tòn tại một đa thức bậc lớn nhất $n$ mà
    \begin{equation}
        d_n = \left | f(x) - P_n(x) \right |_{sn}
    \end{equation}
    đạt được giá trị nhỏ nhất trên tất cả các đa thức.
\end{theorem}

Một nhận xét: hàm $f(x)$ không nhất thiết phải liên tục. Hơn nữa, định lý trên không cho một ước lượng về độ lớn của $d_n$. 

Về các kết quả về tính duy nhất, ta cần độ đo của độ lệch chuẩn nên nghiêm ngặt (strict). Bằng cách định nghĩa một chuẩn $\left | .\right |_{sn}$ là nghiêm ngặt nếu
\begin{equation}
    \left | f+g\right |_{sn} = \left | f\right |_{sn} + \left | g\right |_{sn}
\end{equation}
dẫn đến tồn tại các hằng $\alpha, \beta$ mà $\left | \alpha\right | + \left | \beta\right | \ne 0$ và 
\begin{equation}
    \alpha f(x) + \beta g(x) \equiv 0
\end{equation}

Ta có định lý sau. 
\begin{theorem}
    Giả thiết của định lý \ref{theorem:3.0.0} thêm vào yêu cầu $\left | .\right |_{sn}$ là nghiêm ngặt. Thì đa thức nhỏ nhất, gọi là $P_n(x)$ là duy nhất.
\end{theorem}

\section{Định lý xấp xỉ Weierstrass và Đa thức Bernstein}

Định lý xấp xỉ Weierstrass được phát biểu như sau:
\begin{theorem}
    \label{theorem:weierstrass_approx}
    Gọi $f(x)$ là bất kỳ hàm liên tục nào trong khoảng (đóng) $[a, b]$. Thì với bất kỳ $\epsilon > 0$, tồn tại một số nguyên $n = n(\epsilon)$ và một đa thức $P_n(x)$ với bậc cao nhất $n$ thỏa
    \begin{equation}
        \left | f(x) - P_n(x)\right | < \epsilon
    \end{equation}
    với mọi $x \in [a, b]$
\end{theorem}

Định lý \ref{theorem:weierstrass_approx} đảm bảo có thể xấp xỉ đa thức gần thông qua một khoảng giới hạn đóng chỉ với điều kiện là hàm được xấp xỉ là liên tục. Phát biểu của định lý là về sự tồn tại và không cho một gợi ý nào về cách xây dựng những xấp xỉ. Tuy nhiên, một chứng minh đơn giản và tao nhã của kết quả này do Bernstein trình bày trong định lý

Đa thức Bernstein bậc $n$ cho hàm $f(x)$ trên $[0, 1]$ được định nghĩa:
\begin{equation}
    B_n(f; x) \equiv \sum_{j = 0}^nf(x_j)\beta_{n, j}(x)
\end{equation}
với 
\begin{equation}
    \beta_{n, j}(x) = \binom{n}{j}x^j(1-x)^{n-j}
\end{equation}
\begin{theorem}
    Gọi $f(x)$ là bất kỳ hàm liên tục nào được định nghĩa trên $[0, 1]$. Thì với mọi $x \in [0, 1]$, và bất kỳ số nguyên dương $n$ nào,
    \begin{equation}
        \left | f(x) - B_n(f; x)\right | \leq \frac{9}{4}\omega(f;n^{-1/2})
    \end{equation}
    trong đó modulus của tính liên tục của $f(x)$ trong $[0, 1]$ được định nghĩa
    \begin{equation}
        \omega(f; \delta) = \underset{x, x' \in [0, 1], \left | x - x'\right | \leq \delta}{\text{Least-upper-bound}}\left | f(x) - f'(x)\right |
    \end{equation}
\end{theorem}
Định lý xấp xỉ Weierstrass được suy ra nhờ chọn $n$ đủ lớn sao cho $\omega(f;n^{-1/2}) < \frac{4\epsilon}{9}$.

Nếu $f(x)$ thỏa mãn điều kiện Lipschitz, ta dễ dàng tìm được
\begin{coro}
    Gọi $f(x)$ thỏa mãn điều kiện Lipschitz
    \begin{equation}
        \left | f(x) - f(y)\right | \leq \lambda\left | x-y\right |
    \end{equation}
    với mọi $x, y \in [0, 1]$. Thì với mọi $x\in [0, 1]$
    \begin{equation}
        \left | f(x) - B_n(f; x)\right | \leq  \frac{9}{4}\lambda n^{-1/2}.
    \end{equation}
\end{coro}

\section{Xấp xỉ bình phương tối tiểu}

\begin{theorem}
    Với mỗi hàm chính xác $f(x)$, tồn tại một xấp xỉ đa thức bình phương tối tiểu duy nhất bậc tối đa $n$ mà cực tiểu
    \begin{equation}
        \left \| f(x) - Q_n(x) \right \|_2 \equiv \left\{\int_a^b[f(x) - Q_n(x)]^2dx\right\}^{\frac{1}{2}}
    \end{equation}
\end{theorem}

\begin{theorem}
    Hệ số của ma trận $H_{n+1}(a,b)$ là không suy biến.
\end{theorem}

\begin{theorem}
    Gọi $f(x)$ là liên tục trên $[a, b]$ và $Q_n(x), n=0,1,\dots,$ là đa thức xấp xỉ bình phương tối tiểu đến $f(x)$ trên $[a,b]$. Thì
    \begin{equation}
        \lim_{n \rightarrow \infty}J_n \equiv \lim_{n \rightarrow \infty}\int_a^b[f(x) - Q_n(x)]^2dx = 0
    \end{equation}
    và ta có phương trình Parseval
    \begin{equation}
        \int_a^bf^2(x)dx = \sum_{j =0}^{\infty}c_j^2
    \end{equation}
\end{theorem}

\section{Các đa thức của xấp xỉ "tốt nhất"}

Một độ đo của độ lệch chuẩn giữa một hàm $f(x)$ và một đa thức xấp xỉ bậc $n$, $P_n(x) = a_0 + a_1x + \dots + a_nx^n$ là chuẩn cực đại (maximum norm):
\begin{equation}
    \label{eq:max_norm}
    \left \| f(x) - P_n(x) \right \|_{\infty}  \equiv \underset{a\leq x\leq b}{\max}\left | f(x) - P_n(x) \right | \equiv D(f, P_n)
\end{equation}
Khi một đa thức nào đó mà cực tiểu chuẩn này thì được gọi là đa thức của xấp xỉ "tốt nhất".

Phương trình \eqref{eq:max_norm} định nghĩa các hệ số $\{a_i\}$ không tường minh
\begin{equation}
    d(a_0, a_1, \dots, a_n) \equiv \underset{a\leq x\leq b}{\max}\left | f(x) - P_n(x) \right |
\end{equation}

Một đa thức xấp xỉ tốt nhất được đặc trưng bởi một điểm $\Bar{\mathbf{a}}$ trong không gian $(n+1)$ mà $d(\mathbf{a})$ nhỏ nhất. Định lý sau khẳng định sử tồn tại của đa thức như thế.
\begin{theorem}
    Gọi $f(x)$ là một hàm liên tục trong khoảng $[a,b]$. Thì với bất kỳ số nguyên $n$ nào, tồn tại một đa thức $\hat{P}_n(x)$, bậc tối đa $n$, mà cực tiểu được chuẩn:
    \begin{equation}
        \left \| f(x) - P_n(x) \right \|_{\infty}
    \end{equation}
\end{theorem}

\section{Xấp xỉ lượng giác}

Ta nói $S_n(x)$ là một tổng lượng giác bậc tối đa $n$, nếu
\begin{equation}
    \label{eq:trigonometric_sum}
    S_n(x) = \frac{1}{2}a_0 + \sum_{k=1}^n(a_k\cos(kx) + b_k\sin(kx))
\end{equation}
Bằng cách sử dụng hàm mũ
\begin{align}
    \begin{aligned}
        e^{i\theta} &\equiv \cos\theta + i\sin\theta\\
        \cos\theta &= \frac{1}{2}(e^{i\theta} + e^{-i\theta})\\
        \sin\theta &= \frac{-i}{2}(e^{i\theta} - e^{-i\theta})\\
    \end{aligned}
\end{align}
thì phương trình \eqref{eq:trigonometric_sum} có thể được viết một cách đơn giản hơn
\begin{equation}
    S_n(x) = \sum_{k=-n}^nc_ke^{ikx}
\end{equation}
với 
\begin{align}
    \begin{aligned}
        c_0 &= \frac{a_0}{2}\\
        c_k &= \frac{1}{2}(a_k - ib_k)\\
        c_{-k} &=  \frac{1}{2}(a_k + ib_k)\\
        & k = 1, 2, \dots, n
    \end{aligned}
\end{align}
Kết quả cơ bản trong xấp xỉ bởi tổng lượng giác dựa trên Weierstrass, và có thể được phát biểu thông qua định lý dưới đây:
\begin{theorem}
    Gọi $f(\theta)$ liên tục trên khoảng $[-\pi, \pi]$, và tuân theo chu kỳ $2\pi$. thì với bất kỳ $\epsilon > 0$, tồn tại một $n = n(\epsilon)$ và một tổng lượng giác, $S_n(\theta)$ sao cho
    \begin{equation}
        \left | f(\theta) - S_n(\theta) \right | < \epsilon
    \end{equation}
    với mọi $\theta$.
\end{theorem}

\subsection{Nội suy lượng giác}

\subsection{Xấp xỉ lượng giác bình phương tối tiểu. Chuỗi Fourier}

\subsection{Xấp xỉ lượng giác "tốt nhất"}

Nếu hàm $f(x)$ liên tục trên $[-\pi, \pi]$, ta có thể tìm một tổng lượng giác bậc $n$ mà cực tiểu chuẩn cực đại
\begin{equation}
    \left \| f(x) - S_n(x) \right \|_{\infty}  = \underset{-\pi\leq x\leq \pi}{\max}\left | f(x) - S-n(x) \right |
\end{equation}
Sự tồn tại của đa thức này có thể được xác minh bằng cách sử dụng 


