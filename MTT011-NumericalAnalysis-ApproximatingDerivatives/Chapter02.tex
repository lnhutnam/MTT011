\chapter{KIẾN THỨC NỀN TẢNG}

\section{Đa thức nội suy}

Đa thức nội suy (interpolation polynomial) là một đa thức xấp xỉ mà bằng với hàm mà nó xấp xỉ tại một số điểm cụ thể. Một cách cụ thể, cho trước $n + 1$ điểm phân biệt $x_i, i = 0, \dots, n$, và các giá trị hàm $f(x_i)$ tương ứng, đa thức nội suy với bậc tối đa $n$ cực tiểu chuẩn:
\begin{equation}
    \label{eq:norm_min}
    \left |  f(x) - P_n(x)\right |_{(sn)^{\dagger}} \equiv \sum_{i=0}^n\left |  f(x_i) - P_n(x_i)\right |
\end{equation}
Người ta chỉ ra rằng một đa thức tồn tại và duy nhất; thật vậy, giá trị nhỏ nhất của chuẩn đề cập phía trên là 0. Có hai cách chỉ ra nhận định này là đúng:
\begin{itemize}
    \item Giải hệ phương trình tuyến tính,
    \item Sử dụng dạng Lagrange của đa thức.
\end{itemize}

Xem xét một đa thức có dạng:
\begin{equation}
    \label{eq:lagrange_poly}
    Q_n(x) = \sum_{k=0}^na_kx^k, Q_n(x_i) = f(x_i)
\end{equation}
Bằng cách xem xét các hệ số $a_k$ là ẩn số, ta có một hệ $n+1$ phương trình tuyến tính
\begin{equation}
    A = Q_n(x_i) = a_0 + a_1x_1 + \dots + a_nx_i^{n} = f(x_i), i = 0, 1, \dots, n.
\end{equation}
Nếu hệ số của ma trận là không suy biến, thì hệ có nghiệm duy nhất. Xem xét định thức Vandermonde của ma trận này
\begin{equation}
    \label{eq:lagrange_sys}
    \text{Det}(A) = \prod_{i > j}(x_i - x_j) \equiv \prod_{j = 0}^{n-1}\left[\prod_{i=j+1}^n(x_i - x_j\right]
\end{equation}
Do $\{x_i\}$ là các điểm phân biệt, nên định thức trên không suy biến, và do đó hệ phương trình tuyến tính có nghiệm duy nhất để xác định đa thức nội suy.

Bên cạnh cách vừa đề cập, ta có thể sử dụng dạng Lagrange của đa thức để nhận được trực tiếp đa thức nội suy. Bằng cách đặt
\begin{equation}
    P_n(x) = \sum_{j = 0}^{n}f(x_i)\phi_{n, j}(x)
\end{equation}
trong đó $n+1$ hàm $\phi_{n, j}(x)$ là các đa thức bậc thứ $n$.

Những đa thức như thế được xây dựng một cách dễ dàng, bởi vì $\{x_i\}$ là các điểm phân biệt, tức là, 
\begin{equation}
    \label{eq:zero_points}
    \phi_{n, j}(x) = \frac{(x-x_0)(x-x_1)\cdots(x-x_{j-1})(x-x_{j+1})\cdots(x-x_n)}{(x_j-x_0)(x_j-x_1)\cdots(x_j-x_{j-1})(x_j-x_{j+1})\cdots(x_j-x_n)}, j = 0, 1, \dots, n.
\end{equation}
Các đa thức này được gọi là các hệ số nội suy Lagrange.

Ta đặt
\begin{equation}
    \omega_n(x) \equiv (x-x_0)(x-x_1)\cdots(x-x_{j-1})(x-x_{j+1})\cdots(x-x_n)
\end{equation}
Ta có:
\begin{equation}
   \omega_n'(x) = \left(\frac{\omega_n(x)}{dx}\right)_{x= x_j} 
\end{equation}
Suy ra, các hệ số nội suy Lagrange có thể được viết gọn như sau:
\begin{equation}
    \phi_{n,j}(x) = \frac{\omega_n(x)}{(x-x_j)\omega_n'(x)}
\end{equation}
Bằng cách sử dụng tích trong, dạng Lagrange của đa thức nội suy có thể được viết:
\begin{equation}
    \label{eq:lagrange}
    P_n(x) = \sum_{j = 0}^{n}f(x_i)\prod_{k = 0, k \ne j}\frac{x-x_k}{x_j - x_k}
\end{equation}
Đa thức nội suy Lagrange xác định đa thức được định nghĩa bởi \eqref{eq:lagrange_poly} và \eqref{eq:lagrange_sys} là hệ quả của định lý sau:
\begin{theorem}
    Giả sử $P_n(x)$ và $Q_n(x)$ là hai đa thức bất kỳ, có bậc tối đa $n$, mà
    \begin{equation}
        P_n(x_i) = Q_n(x_i), i = 0, 1, \dots, n,
    \end{equation}
    trong đó $n+1$ điểm $\{x_i\}$ là các điểm phân biệt. Thì
    \begin{equation}
        P_n(x) \equiv Q_n(x)
    \end{equation}
\end{theorem}
Điều này chỉ ra rằng, có một và chỉ một đa thức bậc tối đa $n$ mà \eqref{eq:norm_min} suy biến và được cho bởi \eqref{eq:lagrange} và \eqref{eq:zero_points}

\section{Độ lỗi từng điểm trong nội suy đa thức}

Độ lỗi từng điểm (pointwise error) giữa một hàm, $f(x)$, và một số đa thức xấp xỉ đến nó, $P_n(x)$, được định nghĩa:
\begin{equation}
    R_n(x) \equiv f(x) - P_n(x)
\end{equation}
Với các đa thức nội suy, một biểu diễn hữu ích của $R_n(x)$ dễ dàng có được. Định lý dưới đây phát biểu cho điều này.
\begin{theorem}
    \label{theorem:pointwise_err}
    Giả sử $f(x)$ có đạo hàm đến cấp $(n+1)$, $f^{(n+1)}$, trong một khoảng $[a, b]$. Và $P_n(x)$ là đa thức nội suy cho $f(x)$ tương ứng với $n+1$ điểm phân biệt $x_i, i = 0, 1, \dots, n$ trong đoạn $[a, b]$, tức là $P_n(x_i) = f(x_i)$ và $x_i \in [a, b]$. Thì với mọi $x \in [a, b$, tồn tại một điểm $\xi = \xi(x)$ trong khoảng mở
    \begin{equation}
        \label{eq:pointwise_err_01}
        \min(x_0, x_1, \dots, x_n, x) < \xi < \max(x_0, x_1, \dots, x_n, x)
    \end{equation}
    mà
    \begin{equation}
        \label{eq:pointwise_err_02}
        R_n(x) \equiv f(x) - P_n(x) = \frac{(x-x_0)(x-x_1)\cdots(x-x_n)}{(n+1)!}f^{(n+1)}(\xi) \equiv \frac{\omega_n(x)}{(n+1)!}f^{(n+1)}(\xi)
    \end{equation}
\end{theorem}
Nếu giá trị lớn nhất và giá trị nhỏ nhất của $f^{(n+1)}(x)$ trong $[a, b]$ được xác định, \eqref{eq:pointwise_err_02} cho ta các chặn sai số. Nên lưu ý rằng sai số \eqref{eq:pointwise_err_02} cho các đa thức nội suy giống với phần dư trong khai triển Taylor. Thật vậy, ta có thể giả định một cách đơn giản rằng nếu $\left |  x - x_i\right | < \left |  x - x_0\right |, i = 1, 2, \dots, n$ thì độ lỗi của đa thức nội suy nhỏ hơn độ lỗi trong khai triển Taylor xung quanh điểm $x_0$. Giả định này không phải bao giờ cũng đúng bởi vì $f^{(n+1)}(\xi)$ trong khai triển Taylor và trong \eqref{eq:pointwise_err_02} không được đánh giá tại cùng một điểm $\xi$ với $x$ cho trước.

\section{Đa thức nội suy Newton}

Giả sử $Q_k(x)$ là đa thức nội suy cho $f(x)$, bậc tối đa $k$, tương ứng với $k+1$ điểm phân biệt $x_0, x_1, \dots, x_{k}$. Ta cần tìm các đa thức nội suy, $\{Q_k(x)\}$ với bậc tối đa $k$ mà trong dạng
\begin{equation}
    Q_0(x) \equiv f(x_0)
\end{equation}
và
\begin{equation}
    \label{eq:newton_interpolation_recur}
    Q_k(x) = Q_{k-1}(x) + q_k(x), k = 1, 2, \dots, n,
\end{equation}
trong đó $q_k(x)$ có bậc tối đa $k$.

Do ta cần phải tìm 
\begin{equation}
    Q_k(x_j) = f(x_j) = Q_{k-1}(x_j), j = 0, 1, \dots, k-1
\end{equation}
nên $q_k(x_j) = 0$ tại $k$ điểm này. Dẫn đến, ta thể hiện đa thức tổng quát nhất với bậc tối đa $k$ mà suy biến tại $k$ điểm như nhau:
\begin{equation}
    q_k(x) = a_k \prod_{j=0}^{k-1}(x -x_j)
\end{equation}
Trong phương trình trên, hằng số $a_k$ cần phải được xác định. Để mà $Q_k(x_k) = f(x_k)$, hằng số này phải
\begin{equation}
    a_k = \dfrac{f(x_k) - Q_{k-1}(x_k)}{\prod_{j=0}^{k-1}(x -x_j)}, k = 1, 2, \dots, n,
\end{equation}
Rất tự nhiên, đa thức nội suy bậc không cho điểm khởi tạo $x_0$ là $Q_0(x) \equiv f(x_0)$. Do đó, với $a_0 = f(x_0)$, bằng kỹ thuật đệ quy, ta có đa thức nội suy duy nhất bậc $n$ có dạng như sau:
\begin{equation}
    \label{eq:newton_interpolation_explicit}
    Q_n(x) = a_0 + (x-a_0)a_1 + \dots + (x-x-0)\cdots(x-x_{n-1})a_n
\end{equation}
Hệ số thứ $k$ được gọi là \emph{gia số chia được bậc $k$}, ký hiệu:
\begin{align}
    \begin{aligned}
        a_0 = f[x_0] \\
        a_k = f[x_0, x_1, \dots, x_k], k = 1, 2, \dots
    \end{aligned}
\end{align}
Các giá trị của $f(x)$ mà được nhập vào để xác định $a_k$ là những tham số của $f[x_0, x_1, \dots, x_k]$. Biểu diễn này tương mình hơn dạng đệ quy mà được cho bởi \eqref{eq:newton_interpolation_recur}. Do tính duy nhất của \eqref{eq:newton_interpolation_explicit}, bằng cách sử dụng dạng Lagrange, ta có thể viết:
\begin{equation}
    \label{eq:newton_interpolation_lagrange_form}
    Q_n(x) = \sum_{j = 0}^nf(x_j)\prod_{k=0, k \ne j}^n\frac{x-x_k}{x_j - x_k}
\end{equation}
và hệ số của $x^n$ là
\begin{equation}
    \label{eq:newton_interpolation_lagrange_form_cofficient}
    a_n = f[x_0, x_1, \dots, x_n] = \sum_{j = 0}^n\frac{f(x_j)}{\prod_{k=0, k \ne j}^n(x_j - x_k)}
\end{equation}
Dựa trên dạng \eqref{eq:newton_interpolation_lagrange_form_cofficient}, các \emph{gia số chia được} là các hàm đối xứng theo đối số của chúng. Thật vậy, nếu ta sử dụng ký hiệu truyền thống
\begin{equation}
    f_{i, j, k, \dots} \equiv f[x_i, x_j, x_k, \dots]
\end{equation}
thì tính đối xứng được khai triển như sau
\begin{equation}
    f_{0, 1, \dots, n} = f_{j_0, j_1, \dots, j_n}
\end{equation}
trong đó $(j_0, j_1, \dots, j_n)$ là bất kỳ hoán vị nào của các số nguyên $(0, 1, \dots, n)$.

Ta có thể thu được một dạng tiện lợi hơn \eqref{eq:newton_interpolation_lagrange_form_cofficient} bằng cách sử dụng tính duy nhất của đa thức nội suy. Ta có thể xây dựng đa thức $Q_n(x)$ bằng cách khớp các giá trị của $f(x_j)$ trong một thứ tự nghịch đảo $j = n, n-1, \dots, 1, 0$.
\begin{equation}
    \label{eq:newton_interpolation_explicit_inv}
    Q_n(x) \equiv b_0 + (x - x_n)b_1 + \dots + (x-x_n)(x-x_{n-1})\cdots(x-x_1)b_n
\end{equation}
trong đó $b_k = f[x_n, x_{n-1}, \dots, x_{n-k}$, và $b_0 = f[x_n] = f(x_n)$.

Để ý rằng, $a_n = b_n$, nên từ \eqref{eq:newton_interpolation_explicit} và \eqref{eq:newton_interpolation_explicit_inv}, ta có:
\begin{equation}
0 \equiv [(x-x_0) - (x-x_n)](x-x_1)\cdots(x-x_{n-1})a_n + (a_{n-1}-b_{n-1}x^{n-1} + p_{n-2}(x)    
\end{equation}
trong đó $p_{n-2}(x)$ là một đa thức bậc cao nhất $n-1$.

Và dựa trên tính đối xứng của các gia số chia được, dẫn đến việc
\begin{equation}
    b_{n-1} = f[x_n, x_{n-1}, \dots, x_1] = f[x_1, x_2, \dots, x_n]
\end{equation}
từ $a_n = (a_{n-1} - b_{n-1}) / (x_0 - x_n)$, ta có:
\begin{equation}
    \label{eq:newton_interpolation_ddcoff}
    f[x_0, x_1, \dots, x_n] = \frac{f[x_0, x_1, \dots, x_{n-1}] - f[x_1, x_2, \dots, x_n]}{x_0 - x_n}, n = 1, 2, \dots
\end{equation}

Và bằng cách định nghĩa cho tính hoàn thiện
\begin{equation}
    f[x_0] = f(x_0)
\end{equation}
Đa thức nội suy \eqref{eq:newton_interpolation_explicit} có thể được viết lại như sau:
\begin{equation}
    \label{eq:newton_interpolation_dd}
    Q_n(x) = f[x_0] + (x-x_0)f[x_0, x_1] + \dots + (x-x_0)\cdots(x-x_{n-1})f[x_0, x_1, \dots, x_n]
\end{equation}
Dạng này được gọi là dạng nội suy gia số chia được Newton.

\begin{theorem}
    Giả sử $x, x_0, x_1, \dots, x_{k-1}$ là $k+1$ điểm phân biệt và gọi $f(y)$ có đạo hàm liên tục cấp $k$ trong khoảng
    \begin{equation}
        \min(x, x_0, x_1, \dots, x_{k-1}) < y < \max(x, x_0, x_1, \dots, x_{k-1})
    \end{equation}
    Thì với một số điểm $\xi = \xi(x)$ trong khoảng này
    \begin{equation}
        f[x_0, x_1, \dots, x_{k-1}, x] = \frac{f^{(k)}(\xi)}{(k)!}
    \end{equation}
\end{theorem}

\begin{coro}
    Giả sử
    \begin{equation}
        P_n(x) = \alpha_0 + \alpha_1x + \dots + \alpha_nx^n, \alpha_n \ne 0
    \end{equation}
    là bất kỳ đa thức bậc $n$ nào và gọi $x_0, x_1, \dots, x_{k}$ là $k+1$ điểm phân biệt. Thì
    \begin{equation}
        P_n[x_0, x_1, \dots, x_{k}] = 
        \begin{cases}
            \alpha_n, \text{ nếu } k = 0\\
            0, \text{ nếu } k > n
        \end{cases}
    \end{equation}
\end{coro}

\begin{theorem}
    Giả sử $f(x)$ có đạo hàm liên tục đến cấp $n$ trong khoảng $\min(x_0, x_1, \dots, x_{k}) < y < \max(x_0, x_1, \dots, x_{k})$. Thì nếu những điểm $x_0, x_1, \dots, x_{k}$ phân biệt, 
    \begin{equation}
        f[x_0, x_1, \dots, x_{k}] = \int_0^1dt_1\int_0^{t_1}dt_2\dots\int_0^{t_{n-1}}dt_n \times 
        f^{(n)}(t_n[x_n-x_{n-1}] + \dots + t_1[x_1 - x_0] + x_0)
    \end{equation}
    trong đó, $n \geq 1, t_0 = 1$
\end{theorem}

\begin{coro}
    Giả sử $f^{(n)}(x)$ liên tục trên đoạn $[a, b]$. Với bất kỳ tập điểm $x_0, x_1, \dots, x_k$ trong $[a, b]$ với $k \leq n$, giả sử $f[x_0, x_1, \dots, x_k]$ được cho trước bởi $(10)_k$. Gia số chia được được định nghĩa là một hàm lien6t ục của $k+1$ đối số của nó trong $[a, b]$ và trùng với các định nghĩa ở \eqref{eq:newton_interpolation_lagrange_form_cofficient}, hay \eqref{eq:newton_interpolation_dd} khi các đối số phân biệt nhau.
\end{coro}

\begin{coro}
    Nếu $f^{(n)}(x)$ liên tục trên đoạn $[a, b]$ và $x_0, x_1, \dots, x_n$ trong $[a, b]$ thì 
    \begin{equation}
        f[x_0, x_1, \dots, x_n] = \frac{f^{(n)}(\xi)}{n!},
    \end{equation}
    trong đó 
    \begin{equation}
        \min(x_0, x_1, \dots, x_n) \leq \xi \leq \max(x_0, x_1, \dots, x_n).
    \end{equation}
\end{coro}

\begin{coro}
    Nếu $f^{(n)}(x)$ liên tục trong một lân cận của $x$, thì
    \begin{equation}
        f[\underset{\text{n+1}}{\underbrace{x,x,\dots,x}}] = \frac{f^{(n)}(\xi)}{n!}.
    \end{equation}
\end{coro}
    
\begin{coro}
    Nếu $f^{(n)}(x)$ liên tục trên đoạn $[a, b]$, $y_0, y_1, \dots, y_n$ trong đoạn $[a, b]$, thì
    \begin{equation}
        f[x, y_0, y_1, \dots, y_n] = \frac{f[x, y_1, \dots, y_n]-f[y_0, y_1, \dots, y_n]}{x-y_0}
    \end{equation}
    cho một mở rộng liên tục duy nhất của định nghĩa về gia số chia được.
\end{coro}

\begin{coro}
    Nếu $x_i \ne y_i$ với $0 \leq i \leq p, 0 \leq j \leq q; f^{(m)}(x)$ liên tục trong đoạn $[a, b]; \{x_i\}, \{y_i\}$ trong $[a, b]; 0 \leq p, q\leq m$ thì
    \begin{equation}
        f[x_0, \dots, x_p, y_0, \dots, y_q] = g[x_0, \dots, x_p] = h[y_0, \dots, y_q]
    \end{equation}
    trong đó: $g(x) \equiv f[x, y_0, \dots, y_q$, $h(x) \equiv f[x_0, \dots, x_p, y$,

    cho một mở rộng liên tục duy nhất của định nghĩa của gia số chia được.
\end{coro}

\begin{coro}
    Nếu $f(x)$ có đạo hàm liên tục cấp $m$ trong $[a, b]$; $x_0, \dots, x_p, y_0, \dots, y_q, z_0, \dots, z_r$ trong $[a, b]$; $x_i \ne y_i, x_i \ne z_k, y_j \ne z_k$ với mọi $i, j, k; 0 \leq p,q,r \leq m$; thì
    \begin{equation}
        f[x_0, \dots, x_p, y_0, \dots, y_q, z_0, \dots, z_r] = \left .\frac{1}{p!q!r!}\frac{\partial^p}{\partial x^p}\frac{\partial^q}{\partial y^q}\frac{\partial^r}{\partial z^r}f[x,y,z]\right |_{\xi, \eta, \zeta}
    \end{equation}
    trong đó:
    \begin{align}
        \begin{aligned}
            &\min(x_0, \dots, x_p) \leq \xi \leq \max(x_0, \dots, x_p),\\
            &\min(y_0, \dots, y_q) \leq \eta \leq \max(y_0, \dots, y_q),\\
            &\min(z_0, \dots, z_r) \leq \zeta \leq \max(z_0, \dots, z_r).\\
        \end{aligned}
    \end{align}
\end{coro}

\begin{coro}
    ếu $f^{(m)}(x)$ liên tục trên đoạn $[a, b]$; $x, y, z$ là các điểm phân biệt trong đoạn $[a,b]$; $0 \leq p,q,r \leq m$; thì
    \begin{equation}
        f[\underset{\text{p+1}}{\underbrace{x,x,\dots,x}}\underset{\text{q+1}}{\underbrace{y,y,\dots,y}}\underset{\text{r+1}}{\underbrace{z,z,\dots,z}}] = \frac{1}{p!q!r!}\frac{\partial^p}{\partial x^p}\frac{\partial^q}{\partial y^q}\frac{\partial^r}{\partial z^r}f[x,y,z]
    \end{equation}
\end{coro}

\section{Nội suy tuyến tính tuần tự}

Dạng Newton của đa thức nội suy cho phép dễ dàng tăng độ chính xác cho phương pháp xấp xỉ đa thức. Trong thực hành, các thủ tục lặp được sử dụng và rất hiệu quả khi kết hợp với tính toán máy tính. Bổ đề dưới đây đóng vai trò nền tảng cho các lược đồ \emph{nội suy tuyến tính tuần tự}.

\begin{lemma}
    Giả sử $x_{i_1}, x_{i_2}, \dots, x_{i_n}$ là $n$ điểm phân biệt và $P_{i_1, i_2, \dots, i_n}(x)$ là đa thức nội suy bậc $n-1$ mà thỏa
    \begin{equation}
        P_{i_1, i_2, \dots, i_n}(x_{i_v}) = f(x_{i_v}), v = 1, 2, \dots, n.
    \end{equation}
    Thì nếu $x_j, x_k$ và $x_{i_v}, v = 1, 2, \dots, n$  là bất kỳ $m+2$ điểm phân biệt nào, 
    \begin{equation}
        P_{i_1, i_2, \dots, i_m, j, k}(x) \equiv
        \frac{(x-x_k)P_{i_1, i_2, \dots, i_m, j}(x) - (x-x_j)P_{i_1, i_2, \dots, i_m, k}(x)}{x_j - x_k}, m = 0, 1, 2, \dots
    \end{equation}
\end{lemma}

\section{Số gia tiến}

\subsection{Dạng nội suy trung tâm}

\subsection{Sự phân kỳ}

\section{Một số tính chất giải tích của toán tử gia số}

Một số toán tử cơ bản