\chapter{MỞ ĐẦU}

\section{Động lực nghiên cứu}

Trong nhiều vấn đề thực hành, chúng ta thường gặp các hàm mà các giá trị của nó được biết tại một số điểm nào thông qua các thực nghiệm. Để phục vụ cho một số bài toán nhất định, nhu cầu tính toán tích phân, đạo hàm, hay xấp xỉ giá trị của hàm tại điểm mà ta chưa biết giá trị tại đó là cần thiết. Do đó, nghiên cứu phương pháp xấp xỉ hàm bằng một hàm đã biết mà giá trị tại các điểm đã cho trùng với dữ liệu thực nghiệm là điều cần thiết.

\section{Mục tiêu và đối tượng nghiên cứu của tiểu luận}

Dựa trên bài toán ban đầu - xấp xỉ hàm $f(x)$ bằng một hàm $F(x)$, trùng với $f(x)$ tại các điểm nào đó. Hàm $F(x)$ được xem là nội suy (interpolate) $f(x)$ tại các điểm này. Quá trình xây dựng hàm $F(x)$ nói trên được gọi là phép nội suy (interpolation). Tùy vào bản chất của dữ liệu, loại hàm xấp xỉ được lựa chọn sao cho phù hợp nhất có thể, nhưng đơn giản nhất là \emph{đa thức} bởi vì \emph{mọi hàm liên tục trên một khoảng hữu hạn đều có thể xấp xỉ tốt bằng một đa thức}. Một điều thú vị, do các đa thức và tỉ số của chúng là hàm duy nhất, có thể tính toán được thông qua máy tính. Do đó, đa thức được dùng không chỉ trong nội suy mà còn làm cơ sở cho hầu hết phương pháp trong Giải tích số. 

Trong nhiều ứng dụng, một vấn đề quan trọng là xấp xỉ đạo hàm của một hàm khi biết trước chỉ một số giá trị của hàm. Một phương pháp tiếp cận rõ ràng cho vấn đề này là thiết lập đạo hàm của một đa thức xấp xỉ như một xấp xỉ kỳ vọng đến đạo hàm của hàm. Điều này hoàn toàn có thể thực hiện được cho các đạo hàm cấp cao, nhưng nhìn một cách thật tổng quát, sự xấp xỉ phải tệ đi bởi vì bậc của đạo hàm tăng. 

\section{Ý nghĩa nghiên cứu và thực tiễn ứng dụng}

\section{Cấu trúc của tiểu luận}

Trong tiểu luận này, chúng tôi tập trung làm rõ vấn đề đạo hàm số mà trong tiếp cận xấp xỉ bằng một đa thức nội suy là đối tượng được quan tâm chính. Cấu trúc của tiểu luận được trình bày như sau:
\begin{itemize}
    \item Chương 1: Trình bày động lực nghiên cứu, mục tiêu, đối tượng nghiên cứu chính, ý nghĩa nghiên cứu và thực tiễn ứng dụng của đề tài.
    \item Chương 2: Trình bày kiến thức nền tảng của đề tài.
    \item Chương 3: Trình bày chi tiết về vấn đề quan tâm - đạo hàm số và phương pháp.
    \item Chương 4: Trình bày phương pháp lập trình bài toán đạo hàm số bằng Matlab.
    \item Chương 5: Trình bày các kết luận về đề tài và định hướng với các câu hỏi mở cho nghiên cứu tương lai.
\end{itemize}